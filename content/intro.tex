\chapter*{Introduction} % chapter* je necislovana kapitola
\addcontentsline{toc}{chapter}{Introduction} % rucne pridanie do obsahu
\markboth{Introduction}{Introduction} % vyriesenie hlaviciek

The agile programming principles were stated 20 years ago by the authors of Agile manifesto, but they were used by software developers years before it happened. Experienced programmers develop own skills while doing mistakes and failures on looking for the best approach of development out of numerous possibilities. Such a skills are sound with agile programming principles, which, however, seem to be counter intuitive. Agile methods bring long-term benefits. Using the agile methods while working on the project and seeing its advantages with one's own eyes is a way more convincing. For that reason, it is important to learn agile programming methods. Only theory-based learning is not complete enough for learning agile programming. It requires a lot of practice, especially on projects development.

The term `agile` is more associated with project management rather than its development. This approach on management is known for accepting the fact that everything changes, tending to adapt to new circumstances, and gain from them. Some of the agile development methods are widely used and concerned to be compulsory, the others are mostly taken as extreme and experimental. But all of them make development process efficient, deliberate and responsible. The benefits of these methods are more obvious when it is time to bring changes to the code, and there is no need to `edit and pray`.

The word is full of things, which become stronger when something makes them change. They are not only be naturally evolved systems, but also may be artificial. This feature got name 'antifragile' and was described by Nassim Taleb, author of the series of books on it.

The aim of the Bachelor thesis is to illustrate the point that `agile` is `antifragile` with use of a learning environment. It may used both for acquaintance with the theoretical side of the agile programming methods, and also for creating space to try them out and get feedback instantly. This environment is presented in a web application. It is accompanies with a course, which covers the features suggested by the application and the agile methods.

