\chapter*{Conclusion}  % chapter* je necislovana kapitola
\addcontentsline{toc}{chapter}{Conclusion} % rucne pridanie do obsahu
\markboth{Conclusion}{Conclusion} % vyriesenie hlaviciek

The main aim of the bachelor thesis was to develop the web application, which would introduce and provide environment for practising agile methods of programming. As soon as these methods are initially taken as counter intuitive, it is even more important to try them to understand why they are considered to be efficient and useful. An interactive form enables leading through every step, explaining it and providing feedback.

The resulting web application reached this goal. It is accompanied with a course, which not only illustrates the functionality, but also goes through the theoretical and practical aspects of agile programming methods.

    \subsubsection{Further development}
    Besides the aimed functions of the application, it may be extended with features, which would enable using it at school or university classes, or as a massive open online course.

    Adding an authentication feature would enable adding a feedback and discussions modules. The teachers would be able to track the visits and progress on the course. Also it would be possible for a teachers' team to collaborate on courses. The application enables free passing from one exercise to another for purpose of demonstration. With adding the authentication feature, the teachers would get an option to restrict this feature. A new role of administrator would appear, who would be responsible for registration and courses content management.
    
    The application enables adding different modules for solution estimation. These modules may extend the supported languages and their versions, or suggest another approach of estimation.
    
    The resulting application may become a core another one, which introduces not only aspects of agile programming, but also of agile management.